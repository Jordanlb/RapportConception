\documentclass[11pt]{book}

\usepackage[T1]{fontenc}
\usepackage[frenchb]{babel}

\usepackage{couvresumesPFE} % utilisation du package pfe.sty pour créer la couverture et la page de résumés en français et en anglais

\begin{document}
\pagestyle{empty}
\fontfamily{cmss}
\selectfont

% LES DIFFERENTS CHAMPS DE LA COUVERTURE
\ordre{XXX}  % Le numéro d'ordre donné par le service de la recherche
\auteur{Prénom}{NOM}  % Prénom et nom de l'étudiant
\sautverticalnegatif{1.4} 
% Valeur en cm du saut vertical négatif (ie. vers le haut)
% Modifier cette valeur si le prénom et le nom de l'auteur ne sont pas positionnés au bon endroit par rapport à la ligne "Projet de Fin d'Etudes"
% Si le prénom et le nom tiennent sur la ligne, laisser valeur = 1.4
\specialite{XXX}  % Nom de la spécialité
\anneeuniversitaire{20xx - 20xx}  % Année universitaire
\titre{INTITULE DU PFE : le sujet qui peut s'écrire sur plusieurs lignes}  % sujet du PFE
\entreprisenom{Nom de l'entreprise}  % Nom de l'entreprise
\tuteur{Prénom}{Nom du tuteur}  % Prénom et nom du tuteur en entreprise
\correspondantINSA{Prénom}{Nom du correspondant INSA}  % Prénom et nom du correspondant INSA
\datesout{jj/mm/20xx}  % Date de la soutenance du PFE
\entrepriselogo{(Emplacement du logo de l'entreprise)}{terre}{1} 
% Remplacer le 1er paramètre "(Emplacement du logo de l'entreprise)" par un espace, le texte n'apparaîtra plus 
% Remplacer le 2ème paramètre "insaPFE" par le nom du fichier correspondant au logo de l'entreprise
% Ajuster le 3ème paramètre ici "1" par la largeur souhaitée (en cm) pour l'affichage du logo de l'entreprise
% S'il n'y a pas de logo à afficher, remplacer chacun des 3 paramètres par un espace
%
% RESUMES EN FRANÇAIS ET EN ANGLAIS
\resumefrancais{\textsf{Remplacer ce texte par le résumé en français}}
\resumeanglais{{Change this text by the abstract in english}}
% Les résumés en français et en anglais sont positionnés cote à cote 
% Une hauteur de cadre a été définie pour que les zones réservées aux résumés tiennent sur une seule page : ne pas la modifier 

\makepfe{insa}  % crée la couverture et la page avec les résumés (français et anglais)

\end{document}


